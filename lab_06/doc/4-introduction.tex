\begin{center}
    \textbf{ВВЕДЕНИЕ}
\end{center}
\addcontentsline{toc}{chapter}{ВВЕДЕНИЕ}

Задача коммивояжера является важной и вместе с тем трудноразрешимой [1]. Данная
задача возникает в обширном классе таких приложений, как, например, распознавание траекторий, образов, построение оптимальных схем движения и т.~д.

Задача коммивояжера представляет собой задачу отыскания кратчайшего гамильтонова
пути в полном конечном графе с N вершинами. Все известные методы нахождения точного решения включают в себя поиск пространства решений, которое увеличивается экспоненциально
в зависимости от N. В данной лабораторной работе формулировка задачи ставится следующим образом: «Найти кратчайший незамкнутый маршрут в ориентированном графе, представляющий карту городов России XVI века. При этом раз в 60 суток наступает смена летнего сезона на зимний или наоборот. Зимой можно ходить по рекам в обе стороны за равную цену, летом по течению в 2 раза быстрее, против - в 4 раза медленнее. Необходимо реализовать метод полного перебора и метод на основе муравьиного алгоритма с элитными муравьями.»

\textbf{Цель лабораторной работы} --- выполнить сравнительный анализ метода полного перебора с методом на базе муравьиного алгоритма. Для достижения поставленной цели необходимо выполнить следующие задачи:

\begin{itemize}
	\item[---] описать схемой алгоритма и реализовать метод полного перебора для решения задачи коммивояжера;
	\item[---] описать схемой алгоритма и реализовать метод на основе муравьиного алгоритма для решения задачи коммивояжера;
    \item[---] выполнить оценку трудоемкости по разработанным схемам алгоритмов;
    \item[---] указать преимущества и недостатки реализованных методов;
    \item[---] выполнить сравнительный анализ двух методов решения задачи коммивояжера;
    \item[---] выполнить параметризацию метода на основе муравьиного алгоритма по трем его параметрам и дать рекомендации о комбинациях значений параметров для работы алгоритма.
\end{itemize}