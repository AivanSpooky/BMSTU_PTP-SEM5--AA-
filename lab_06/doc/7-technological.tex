\chapter{Технологическая часть}

В данном разделе будут приведены требования к программному обеспечению, средства реализации, листинги кода.

\section{Требования к программному обеспечению}

Входные данные: квадратная матрица стоимостей и дополнительные параметры к муравьиному алгоритму;

Выходные данные: минимальная стоимость маршрута и сам маршрут (массив посещенных городов).

\section{Средства реализации}
В данной работе для реализации был выбран язык программирования $C\#$ [3]. Выбор обсуловлен наличием $LINQ$---выражений [4], упрощающих работу с коллекциями и данными, а именно для нахождения всех возможных замкнутых маршрутов и для выбора элитных муравьев из всех муравьев. Время было замерено с помощью методов класса \textit{Stopwatch} [5].

\section{Реализация алгоритмов}

В листингах~\ref{lst:lev_recursion} ---~\ref{lst:lev_table} представлены реализации алгоритмов полного перебора и муравьиного.

\clearpage

\begin{center}
\captionsetup{justification=raggedright,singlelinecheck=off}
\begin{lstlisting}[label=lst:lev_recursion,caption=Алгоритм полного перебора,numbers=none]
static void RunBruteForce()
{
    Console.WriteLine("\nBruteForce algo...");
    List<int> bestRoute = null;
    int minCost = int.MaxValue;
    foreach (var permutation in GetPermutations(Enumerable.Range(0, n), n))
    {
        int cost = 0;
        bool validRoute = true;

        for (int i = 0; i < permutation.Count() - 1; i++)
        {
            int from = permutation.ElementAt(i);
            int to = permutation.ElementAt(i + 1);

            if (graph[from, to] == 0)
            {
                validRoute = false;
                break;
            }
            cost += graph[from, to];
        }
        if (validRoute && cost < minCost)
        {
            minCost = cost;
            bestRoute = permutation.ToList();
        }
    }
    if (bestRoute != null)
    {
        Console.WriteLine("Min cost: " + minCost);
        Console.WriteLine("Best route: " +
            string.Join(" -> ", bestRoute.Select(i => cities[i])));
    }
    else
        Console.WriteLine("No route found.");
}
\end{lstlisting}
\end{center}
\clearpage

\begin{center}
\captionsetup{justification=raggedright,singlelinecheck=off}
\begin
{lstlisting}[label=lst:lev_table,caption=Муравьиный алгоритм,numbers=none]
public List<int> Run()
{
    for (int t = 0; t < tMax; t++)
    {
        List<List<int>> tours = new List<List<int>>();
        List<double> lengths = new List<double>();

        for (int k = 0; k < numAnts; k++)
        {
            List<int> tour = ConstructSolution();
            double length = CalculateTourLength(tour);
            tours.Add(tour);
            lengths.Add(length);

            if (length < BestLength)
            {
                BestLength = length;
                BestTour = new List<int>(tour);
            }
        }

        UpdatePheromones(tours, lengths);

        seasonCounter++;
        if (seasonCounter % 60 == 0)
        {
            UpdateSeason();
            isSummer = !isSummer;
        }
    }
    return BestTour;
}

List<int> ConstructSolution()
{
    List<int> tour = new List<int>();
    bool[] visited = new bool[n];
    int currentCity = rand.Next(n);
    tour.Add(currentCity);
    visited[currentCity] = true;
    while (tour.Count < n)
    {
        int nextCity = SelectNextCity(currentCity, visited);
        if (nextCity == -1)
            break;
        tour.Add(nextCity);
        visited[nextCity] = true;
        currentCity = nextCity;
    }
    return tour;
}

int SelectNextCity(int currentCity, bool[] visited)
{
    double[] probabilities = new double[n];
    double sum = 0.0;
    for (int i = 0; i < n; i++)
    {
        if (!visited[i] && graph[currentCity, i] > 0)
        {
            probabilities[i] = Math.Pow(pheromone[currentCity, i], alpha) *
                               Math.Pow(1.0 / graph[currentCity, i], beta);
            sum += probabilities[i];
        }
        else
            probabilities[i] = 0.0;
    }
    if (sum == 0.0)
        return -1;
    double r = rand.NextDouble() * sum;
    double total = 0.0;
    for (int i = 0; i < n; i++)
    {
        total += probabilities[i];
        if (total >= r)
            return i;
    }

    return -1;
}

void UpdatePheromones(List<List<int>> tours, List<double> lengths)
{
    for (int i = 0; i < n; i++)
        for (int j = 0; j < n; j++)
            pheromone[i, j] *= (1.0 - rho);

    for (int k = 0; k < tours.Count; k++)
    {
        double delta = 1.0 / lengths[k];
        List<int> tour = tours[k];

        for (int i = 0; i < tour.Count - 1; i++)
        {
            int from = tour[i];
            int to = tour[i + 1];
            pheromone[from, to] += delta;
        }
    }

    var eliteAnts = lengths.Select((length, index) => new { length, index })
                            .OrderBy(l => l.length)
                            .Take(numEliteAnts)
                            .Select(l => tours[l.index]);
    foreach (var tour in eliteAnts)
    {
        double delta = 1.0 / CalculateTourLength(tour);
        for (int i = 0; i < tour.Count - 1; i++)
        {
            int from = tour[i];
            int to = tour[i + 1];
            pheromone[from, to] += delta;
        }
    }
}

double CalculateTourLength(List<int> tour)
{
    double length = 0.0;
    for (int i = 0; i < tour.Count - 1; i++)
        length += graph[tour[i], tour[i + 1]];
    return length;
}

void UpdateSeason()
{
    if (isSummer)
    {
        for (int i = 0; i < n; i++)
            for (int j = 0; j < n; j++)
                graph[i, j] = originalGraph[i, j];
    }
    else
    {
        for (int i = 0; i < n; i++)
            for (int j = 0; j < n; j++)
                if (i != j)
                    if ((i + j) % 2 == 0)
                        graph[i, j] = originalGraph[i, j] / 2;
                    else
                        graph[i, j] = originalGraph[i, j] * 4;
    }
}
\end{lstlisting}
\end{center}

\vspace{5mm}

\textbf{ВЫВОД}

В данном разделе были рассмотрены требования к программному обеспечению, используемые средства реализации, а также приведены листинги кода двух алгоритмов решения задачи коммивояжера.

\clearpage