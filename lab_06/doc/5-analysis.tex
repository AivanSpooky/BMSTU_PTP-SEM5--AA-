\chapter{Аналитическая часть}

В данном разделе будут рассмотрены два метода решения задачи коммивояжера, основанные на полном переборе и муравьином алгоритме.

\section{Алгоритм полного перебора}

Алгоритм полного перебора (АПП) осуществляет поиск в пространстве $N!$ решений посредством перебора всех вариантов маршрута и выбирает наикратчайший из них. Результатом работы алгоритма является точное решение. Недостатком АПП является его временная сложность --- пространство поиска растёт факториально.

\section{Муравьиный алгоритм}

В основе муравьиного алгоритма лежит идея моделирования поведения колонии муравьев. Каждый муравей определяет свой маршрут на основе оставленных другими муравьями феромонов, а также сам оставляет оставляет феромоны, чтобы последующие муравьи ориентировались по ним. В результате при прохождении каждым муравьем своего маршрута наибольшее число феромонов остается на самом оптимальном пути. Временная сложность алгоритма была оценена как $683 - (42,467 · N) + (1,0696 · N^2)$ [2]. Однако главный недостаток алгоритма заключается в том, что, по сравнению с алгоритмом полного перебора, он даёт приближенное решение задачи, а не точное.

\vspace{5mm}

\textbf{ВЫВОД}

В данном разделе были рассмотрены два основных алгоритма, используемые в методах решения задачи коммивояжера.