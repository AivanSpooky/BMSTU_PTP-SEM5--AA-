\begin{center}
    \textbf{ВВЕДЕНИЕ}
\end{center}
\addcontentsline{toc}{chapter}{ВВЕДЕНИЕ}

В современном программировании эффективная обработка данных играет ключевую роль в повышении производительности приложений. Одним из подходов к оптимизации выполнения задач является конвейерная обработка ($pipeline$ $processing$), которая позволяет разделить процесс обработки данных на последовательные стадии. Каждая стадия выполняется независимо, что обеспечивает параллельную обработку и более эффективное использование системных ресурсов.

\textbf{Цель лабораторной работы} --- получение навыка организации параллельных вычислений по конвейерному принципу. Для достижения поставленной цели необходимо выполнить следующие задачи:

\begin{itemize}
	\item[---] описать входные, выходные данные, а также преобразования входных данных в выходные;
	\item[---] реализовать алгоритм для парсинга рецептов с веб-сайта с использованием конвейерной обработки и многопоточности на основе нативных потоков;
	\item[---] протестировать разработанный алгоритм по методологии черного ящика;
    \item[---] описать пример работы программы на конкретном случае;
    \item[---] сделать выводы, основываясь на информации в полученных лог-файлах.
\end{itemize}