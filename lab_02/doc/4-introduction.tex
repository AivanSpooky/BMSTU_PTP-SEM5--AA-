\begin{center}
    \textbf{ВВЕДЕНИЕ}
\end{center}
\addcontentsline{toc}{chapter}{ВВЕДЕНИЕ}

Прямоугольной матрицей называется совокупность чисел, расположенных в виде прямоугольной таблицы, содержащей \textit{n} строк и \textit{m} столбцов.

\textbf{Цель лабораторной работы} --- выполнить оценки ресурсной эффективности алгоритмов умножения матриц и их реализации. Для достижения поставленной цели необходимо выполнить следующие задачи:

\begin{itemize}
	\item[---] описать математическую основу стандартного алгоритмов и алгоритма Винограда умножения матриц;
	\item[---] описать модель вычислений;
	\item[---] разобрать алгоритмы умножения матриц (стандартный, Винограда, оптимизированный алгоритм Винограда;
    \item[---] выполнить оценку трудоемкости разрабатываемых алгоритмов;
    \item[---] реализовать разработанные алгоритмы в программном обеспечении с 2 режимами работы (одиночного расчета и массированного замера процессорного времени выполнения реализации каждого алгоритма);
    \item[---] выполнить замеры процессорного времени выполнения реализации разработанных алгоритмов в зависимости от варьируемого линейного размера матриц;
    \item[---] выполнить сравнительный анализ рассчитанных трудоемкостей и результатов замера процессорного времени выполнения реализации трех алгоритмов с учетом лучшего и худшего случаев по трудоемкости
\end{itemize}