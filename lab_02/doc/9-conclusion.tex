\begin{center}
    \textbf{ЗАКЛЮЧЕНИЕ}
\end{center}
\addcontentsline{toc}{chapter}{ЗАКЛЮЧЕНИЕ}

В результате исследования было определено, что классический алгоритм умножения матриц проигрывает по времени алгоритму Винограда примерно в 1.2 раза из-за того, что в алгоритме Винограда часть вычислений происходит заранее, а также сокращается часть сложных операций - операций умножения, поэтому предпочтение следует отдавать алгоритму Винограда. Но лучшие показатели по времени выдает оптимизированный алгоритм Винограда -- он примерно в 1.2 раза быстрее алгоритма Винограда на размерах матриц свыше 10 из-за замены операций равно и плюс на операцию плюс-равно, за счет замены операции умножения операцией сдвига, а также за счет предвычислений некоторых слагаемых, что дает проводить часть вычислений быстрее. Поэтому при выборе самого быстрого алгоритма предпочтение стоит отдавать оптимизированному алгоритму Винограда.

\vspace{5mm}

В ходе выполнения данной лабораторной работы были решены следующие задачи:
\begin{itemize} \item[---] описана математическая основа стандартного алгоритма и алгоритма Винограда умножения матриц; \item[---] представлена модель вычислений для анализа трудоемкости алгоритмов; \item[---] разобраны алгоритмы умножения матриц: стандартный, Винограда и оптимизированный алгоритм Винограда; \item[---] выполнена оценка трудоемкости разработанных алгоритмов; \item[---] реализованы алгоритмы в программном обеспечении с двумя режимами работы: одиночный расчет и массированный замер процессорного времени выполнения каждого алгоритма; \item[---] выполнены замеры процессорного времени выполнения реализации разработанных алгоритмов в зависимости от варьируемого линейного размера матриц; \item[---] проведен сравнительный анализ рассчитанных трудоемкостей и результатов замеров процессорного времени выполнения трех алгоритмов, с учетом лучшего и худшего случаев по трудоемкости. \end{itemize}