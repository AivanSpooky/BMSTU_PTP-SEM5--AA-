\chapter{Технологическая часть}

В данном разделе будут приведены требования к программному обеспечению, средства реализации, листинги кода.

\section{Требования к программному обеспечению}

Входные данные: две матрицы, где количество столбцов первой матрицы равна количетсву строк второй матрицы;

Выходные данные: матрица, где количество строк равно количеству строк первой матрицы, а количество столбцов равно количеству столбцов второй матрицы.

\section{Средства реализации}
В данной работе для реализации был выбран язык программирования $C++$ [3]. Выбор обсуловлен наличием функции вычисления процессорного времени в библиотеке \textit{std::chrono} [4]. Время было замерено с помощью функции \textit{std::clock} [5].

\section{Реализация алгоритмов}

В листингах \ref{lst:lev_recursion} - \ref{lst:dam_lev_dyn} представлены реализации алгоритмов нахождения расстояния Левенштейна и Дамерау–Левенштейна.

\clearpage

\begin{center}
\captionsetup{justification=raggedright,singlelinecheck=off}
\begin{lstlisting}[label=lst:lev_recursion,caption=Классический алгоритм]
Matrix Simple(const Matrix& m1, const Matrix& m2)
{
    size_t rows1 = m1.rows();
    size_t cols1 = m1.columns();
    size_t cols2 = m2.columns();

    Matrix res(rows1, cols2, 0);

    for (size_t i = 0; i < rows1; ++i)
        for (size_t j = 0; j < cols2; ++j)
            for (size_t k = 0; k < cols1; ++k)
                res[i][j] = res[i][j] + m1[i][k] * m2[k][j];

    return res;
}
\end{lstlisting}
\end{center}


\begin{center}
\captionsetup{justification=raggedright,singlelinecheck=off}
\begin
{lstlisting}[label=lst:lev_table,caption=Алгоритм Винограда]
Matrix Winograd(const Matrix& m1, const Matrix& m2)
{
    size_t rows1 = m1.rows();
    size_t cols1 = m1.columns();
    size_t cols2 = m2.columns();

    Matrix res(rows1, cols2);

    std::vector<int> row_factors(rows1, 0);
    std::vector<int> col_factors(cols2, 0);

    for (size_t i = 0; i < rows1; ++i)
        for (size_t j = 0; j < cols1 / 2; ++j)
            row_factors[i] = row_factors[i] + m1[i][2 * j] * m1[i][2 * j + 1];

    for (size_t i = 0; i < cols2; ++i)
        for (size_t j = 0; j < cols1 / 2; ++j)
            col_factors[i] = col_factors[i] + m2[2 * j][i] * m2[2 * j + 1][i];

    for (size_t i = 0; i < rows1; ++i)
    {
        for (size_t j = 0; j < cols2; ++j)
        {
            res[i][j] = -row_factors[i] - col_factors[j];
            for (size_t k = 0; k < cols1 -1; k += 2)
            {
                res[i][j] = res[i][j] + (m1[i][k] + m2[k + 1][j]) *
                    (m1[i][k + 1] + m2[k][j]);
            }
        }
    }

    if (cols1 % 2)
    {
        for (size_t i = 0; i < rows1; ++i)
            for (size_t j = 0; j < cols2; ++j)
                res[i][j] = res[i][j] + m1[i][cols1 - 1] *
                m2[cols1 - 1][j];
    }

    return res;
}
\end{lstlisting}
\end{center}

\begin{center}
\captionsetup{justification=raggedright,singlelinecheck=off}
\begin{lstlisting}[label=lst:dam_lev_dyn,caption=Оптимизированный алгоритм Винограда]
Matrix WinogradOpt(const Matrix& m1, const Matrix& m2)
{
    size_t rows1 = m1.rows();
    size_t cols1 = m1.columns();
    size_t cols2 = m2.columns();

    Matrix res(rows1, cols2);

    std::vector<int> row_factors(rows1, 0);
    std::vector<int> col_factors(cols2, 0);

    for (size_t i = 0; i < rows1; ++i)
        for (size_t j = 0; j < cols1 / 2; ++j)
            row_factors[i] += m1[i][2 * j] * m1[i][2 * j + 1];

    for (size_t i = 0; i < cols2; ++i)
        for (size_t j = 0; j < cols1 / 2; ++j)
            col_factors[i] += m2[2 * j][i] * m2[2 * j + 1][i];

    for (size_t i = 0; i < rows1; ++i)
    {
        for (size_t j = 0; j < cols2; ++j)
        {
            res[i][j] = -row_factors[i] - col_factors[j];
            for (size_t k = 0; k < cols1 - 1; k += 2)
            {
                res[i][j] += (m1[i][k] + m2[k + 1][j]) *
                    (m1[i][k + 1] + m2[k][j]);
            }

            if (cols1 % 2)
            {
                res[i][j] += m1[i][cols1 - 1] * m2[cols1 - 1][j];
            }
        }
    }

    return res;
}
\end{lstlisting}
\end{center}

\vspace{5mm}

\textbf{ВЫВОД}

В данном разделе были рассмотрены требования к программному обеспечению, используемые средства реализации, а также приведены листинги кода для умножения матриц с помощью классического алгоритма, алгоритма Винограда, оптимизированного алгоритма Винограда.

\clearpage