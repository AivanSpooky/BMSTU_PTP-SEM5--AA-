\chapter{Аналитическая часть}

В данном разделе будут рассмотрены алгоритмы нахождения расстояний Левенштейна и Дамерау-Левенштейна.

\section{Описание алгоритмов}

Расстояние Левенштейна ~---~ это минимальное число односимвольных преобразований (удаления, вставки или замены), необходимых для превращения одной строки в другую [2].

\subsection{Расстояние Левенштейна}
Для двух строк $S_{1}$ и $S_{2}$, представленных в виде списков символов, длинной M и N соответственно, расстояние Левенштейна рассчитывается по рекуррентной формуле \ref{eq:L}:

\begin{equation}
	\label{eq:L}
	D(S_{1}, S_{2}) = \begin{cases}
	len(S_{1}), &\text{если $len(S_{2})=0$}\\
	len(S_{2}), &\text{если $len(S_{1})=0$}\\
	D(tail(S_{1}), tail(S_{2})), &\text{если $head(S_{1})=head(S_{2})$}\\
	1 + min \begin{cases}
		D(tail(S_{1}), S_{2}),\\
		D(S_{1}, tail(S_{2})),\\
		D(tail(S_{1}), tail(S_{2})),\\
	\end{cases}
	&\text{иначе}
	\end{cases}
\end{equation}

\vspace{5mm}

где:
\begin{itemize}
	\item $len(S)$ — длина списка $S$;
	\item $head(S)$ — первый элемент списка $S$;
	\item $tail(S)$ — список $S$ без первого элемента;
\end{itemize}


\subsection{Расстояние Дамерау-Левенштейна}
В алгоритме поиска расстояния Дамерау-Левенштейна, помимо вставки, удаления, и замены присутствует операция перестановки символов.
Расстояние Дамерау-Левенштейна может быть вычисленно по рекуррентной формуле \ref{eq:DL}:

\begin{equation}
	\label{eq:DL}
	D(S1, S2) = \begin{cases}
	len(S_{1}), \hspace{4cm} \text{если $len(S_{2})=0$}\\
	len(S_{2}), \hspace{4cm} \text{если $len(S_{1})=0$}\\
    D(tail(S_{1}), tail(S_{2})), \hspace{1.25cm} \text{если $head(S_{1})=head(S_{2})$}\\
	1 + min \begin{cases}
		D(tail(S1), S2),\\
		D(S1, tail(S2)),\\
		D(tail(S1), tail(S2)),\\
	\end{cases}\\
	1 + min \begin{cases}
		D(tail(tail(S1)), tail(tail(S2))), &\text{(*)}\\
		D(tail(S1), S2),\\
		D(S1, tail(S2)),\\
        D(tail(S1), tail(S2)),\\
	\end{cases}
	\end{cases} 
\end{equation}
 (*): если head(S1) = head(tail(S2)) и head(S2) = head(tail(S1));


\textbf{ВЫВОД}


В данном разделе были рассмотрены два основных алгоритма для вычисления расстояний между строками: расстояние Левенштейна и расстояние Дамерау-Левенштейна.