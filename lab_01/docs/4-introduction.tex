\begin{center}
    \textbf{ВВЕДЕНИЕ}
\end{center}
\addcontentsline{toc}{chapter}{ВВЕДЕНИЕ}

Расстояние Левенштейна и Дамерау-Левентшейна представляет собой минимальное количество операций (вставка, удаление и замена символов), требуемое для преобразования одной строки в другую. Расстояние Левенштейна используется для исправления ошибок в словах, поиска дубликатов текстов, сравнения геномов и прочих полезных операций с символьными последовательностями [1].

\textbf{Цель лабораторной работы} --- сравнение алгоритмов нахождения расстояния Левенштейна и Дамерау-Левенштейна. Для достижения поставленной цели необходимо выполнить следующие задачи:

\begin{itemize}
	\item реализовать указанные алгоритмы поиска расстояний (один алгоритм с использованием рекурсии, два алгоритма с использованием динамического программирования);
	\item проанализировать рекурсивную и матричную реализации алгоритмов по затраченному процессорному времени и памяти на основе экспериментальных данных;
	\item описать и обосновать полученные результаты в отчете.
\end{itemize}