\begin{center}
    \textbf{ВВЕДЕНИЕ}
\end{center}
\addcontentsline{toc}{chapter}{ВВЕДЕНИЕ}

В данной лабораторной работе исследуется применение графовых моделей для анализа программного кода, реализующего конвейерную многопоточную обработку рецептов с веб-сайта. Использование таких моделей позволяет наглядно отразить информационные и управляющие зависимости, упростить оптимизацию и понимание структуры программы.

\textbf{Цель лабораторной работы} --- получение навыка построения и применения графовых моделей для анализа кода. Для достижения поставленной цели необходимо решить следующие задачи:

\begin{itemize}
	\item[---] построены информационный граф, информационная история, граф управления и операционная история для выбранного фрагмента кода;
	\item[---] проанализированы построенные графовые модели;
    \item[---] сделан вывод о применимостиграфовых моделей к задаче анализа программного кода.
\end{itemize}