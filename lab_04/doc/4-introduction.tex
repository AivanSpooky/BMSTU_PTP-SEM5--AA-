\begin{center}
    \textbf{ВВЕДЕНИЕ}
\end{center}
\addcontentsline{toc}{chapter}{ВВЕДЕНИЕ}

В общем случае поток исполнения представляет собой последовательность инструкций, выполняемых на выделенном процессорном ядре и управляемых планировщиком операционной системы. Потоки могут быть приостановлены или заблокированы в процессе выполнения. Они создаются внутри процесса и совместно используют его ресурсы, такие как оперативная память и дескрипторы файлов. Такой механизм организации потоков называется нативными потоками [1]. Нативные потоки обеспечивают эффективное использование системных ресурсов и позволяют выполнять несколько задач параллельно в рамках одного процесса, что существенно повышает производительность приложений.

\textbf{Цель лабораторной работы} --- сравнить основные принципы последовательных вычислений с параллельными на основе нативных потоков. Для достижения поставленной цели необходимо выполнить следующие задачи:

\begin{itemize}
	\item[---] описать входные, выходные данные, а также преобразования входных данных в выходные;
	\item[---] реализовать два алгоритма для загрузки контента из $HTML$---страниц: последовательный и параллельный с использованием нативных потоков;
	\item[---] протестировать разработанные алгоритмы по методологии черного ящика;
    \item[---] описать пример работы программы на конкретном случае;
    \item[---] выполнить сравнительный анализ последовательного и параллельного алгоритма по времени выполнения в зависимости от количества обрабатываемых страниц.
\end{itemize}