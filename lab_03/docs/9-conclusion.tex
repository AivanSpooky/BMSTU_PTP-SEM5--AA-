\begin{center}
    \textbf{ЗАКЛЮЧЕНИЕ}
\end{center}
\addcontentsline{toc}{chapter}{ЗАКЛЮЧЕНИЕ}

Исследование показало, что бинарный поиск демонстрирует значительно более высокую эффективность по сравнению с методом полного перебора. Трудоемкость бинарного поиска составляет $O(\log N)$, что делает его предпочтительным выбором для поиска элементов в отсортированных массивах, особенно при больших объемах данных.

Полный перебор, хотя и является простым в реализации, имеет линейную трудоемкость $O(N)$, что приводит к значительным затратам времени, особенно в худшем случае, когда элемент расположен в конце массива или отсутствует.

\vspace{5mm}

В ходе выполнения данной лабораторной работы были решены следующие задачи:
\begin{itemize}
    \item проведена оценка трудоемкости разрабатываемых алгоритмов;
    \item реализованы алгоритмы поиска значений в словаре;
    \item выполнен сравнительный анализ сложности двух алгоритмов на основе замеров количества сравнений для каждого элемента массива и анализ данных лучшего и худшего случаев.
\end{itemize}