\chapter{Технологическая часть}

В данном разделе будут приведены требования к программному обеспечению, средства реализации, листинги кода.

\section{Требования к программному обеспечению}

Входные данные: массив и искомое значение;

Выходные данные: индекс найденного значения и количество сравнений.

\section{Средства реализации}
В данной работе для реализации был выбран язык программирования $C\#$ с использованием платформы $.NET\ 8$ [1]. Выбор обусловлен наличием:
\begin{itemize}
    \item возможности создавать массивы определенного размера: $int[N]$;
    \item встроенной поддержки $Windows Forms$, что позволяет создавать пользовательские интерфейсы и визуализировать данные. В частности, для построения гистограмм были использованы средства, представленные в пространстве имен $System.Windows.Forms.DataVisualization.Charting$ [2].
\end{itemize}

\section{Реализация алгоритмов}

В листингах \ref{lst:fullsearch} - \ref{lst:binsearch} представлены реализации алгоритмов.

\clearpage

\begin{center}
\captionsetup{justification=raggedright,singlelinecheck=off}
\begin{lstlisting}[label=lst:fullsearch,caption=Алгоритм использующий полный перебор]
private int FullSearch(int[] array, int target, out int comparisons)
{
    comparisons = 0;
    for (int i = 0; i < N; i++)
    {
        comparisons++;
        if (array[i] == target)
            return i;
    }
    return -1;
}
\end{lstlisting}
\end{center}


\begin{center}
\captionsetup{justification=raggedright,singlelinecheck=off}
\begin{lstlisting}[label=lst:binsearch,caption=Алгоритм использующий полный перебор]
private int BinarySearch(int[] array, int target, out int comparisons)
{
    comparisons = 0;
    int left = 0, right = N - 1;

    while (left <= right)
    {
        comparisons++;
        int mid = (left + right) / 2;
        if (array[mid] == target)
            return mid;
        if (array[mid] < target)
            left = mid + 1;
        else
            right = mid - 1;
    }
    return -1;
}
\end{lstlisting}
\end{center}

\clearpage

\vspace{5mm}

\textbf{ВЫВОД}

В данном разделе были реализованы алгоритмы поиска заданного значения в массиве полным перебором и с помощью двоичного поиска, рассмотрены средства реализации, предусмотрены требования к программному обеспечению.

\clearpage